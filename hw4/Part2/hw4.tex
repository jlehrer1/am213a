\documentclass{article}
\usepackage[utf8]{inputenc}
\setlength{\parindent}{0pt} 
\usepackage{amssymb}
\usepackage{amsmath}
\newcommand{\N}{\mathbb{N}}
\newcommand{\Z}{\mathbb{Z}}
\newcommand{\Q}{\mathbb{Q}}
\newcommand{\R}{\mathbb{R}}
\newcommand{\C}{\mathbb{C}}
\newcommand{\ra}{\longrightarrow}

\title{Homework 4: Theory Questions}
\date{}
\author{Julian Lehrer}
\begin{document}
\maketitle

\textbf{Question 1.} Lemma 1: $A^n = QU^n Q^*$ where $U$ is upper triangular and $Q$ is unitary. Proof (induction on $n$). For $n=1$, we have that $A = QUQ^*$ by the Schur decomposition. Then suppose $A^n = QU^n Q^*$, and show $A^{n+1}=QU^{n+1}Q^*$. We have that $A^{n+1} = AA^{n} = (Q U Q^*)(QU^n Q^*) = QU Q^* Q U^n Q^* = QUIU^n Q^* = QU^{n+1}Q^*$, as desired. Since $A^n$ is similar to $QU^n Q^*$, we have that the spectrum of $A^n$ is the same as the spectrum of $U^n$. Since all norms are equivalent in a finite vector space, the result generalize for any matrix norm. \\

Proof $(\implies)$. Consider $\|A \|_F$, the Frobenius norm given by $\sqrt{\sum_{i}\sum_{j}|a_{ij}|}$. If $\sqrt{\sum_{i}\sum_j|a^n_{ij}|} \ra 0$ as $n \ra \infty$, then we must have that $|a_{ij}| < 0$, since each entry of the matrix must go to zero. Then since $\|A^n\|_F \ra 0 \iff \|U^n\|_F \ra 0$ as $n \ra \infty$ (from Lemma 1), and the diagonals of $U$ contain the eigenvalues of $A$, we have by necessity that $p(A) < 1$ since $|u_{ij}| < 1$. \\

Proof $(\impliedby)$. Suppose that $p(A) < 1$. Then $p(U) < 1$, and since the diagonal elements of $U$ are the spectra of $A$ we have that $|u_{ii}| < 1$. Intuition: If each eigenvalue is less than one, and the determinant is the product of the eigenvalues, then $|det(A)| < 1$. Therefore, the vector mapped under $A$ is compressed, since the determinant measures the change of volume under the image of $A$. Therefore, upon repeated application of a compressive map, the volume goes to zero. Since matrix norms measure how stretched a vector becomes under the image of $A$, it follows that the norm collapses to zero. \\ 

Proof (attempt). Since $|u_{ii}| < 1$, we have that by the Gershgorin circle theorem, all eigenvalues are within the circles centered at $u_{ii}$ with radius $r_i = \sum_{i, i \neq j} u_{ij}$. Since we also know that the eigenvalues $u_{ii}$ have absolute values less than 1, it follows that the radii of the Gershgorin circles must be less than $1-u_{ii}$. But this implies that $r_i \leq 1$, so $|u_{ij}| < 1$. Therefore, the 1 norm $\max_{i} |\sum_{i=1}^m |u_{ij}| < 1$, so $\|U\| < 1$. Then since $\max_{i} |\sum_{i=1}^m |u_{ij}|^2 < \max_{i} |\sum_{i=1}^m |u_{ij}| < 1$, we have that $\|U^2\|_1 < \|U\|_1 < 1$. Therefore, $\|U^n\|_1 \ra 0$ as $n \ra \infty$. \\

\textbf{Question 2.} Lemma 1: 
\begin{equation*}
    \text{det}\begin{pmatrix}
        A&0\\
        B&C
    \end{pmatrix} = \text{det}(A)\text{det}(C) = \text{det}\begin{pmatrix}
        A&B\\
        0&C
    \end{pmatrix}
\end{equation*} 
Proof: I referenced https://www.statlect.com/matrix-algebra/determinant-of-block-matrix. I wasn't fully able to prove this myself, although I learned it in undergraduate linear algebra. \\

Now, note that if two matrices have the same characteristic polynomial, then necessarily they have the same eigenvalues. Therefore, consider 
\begin{equation*}
    \det \begin{pmatrix}
        AB-\lambda I&0\\
        B & -\lambda I
    \end{pmatrix} = \det(AB-\lambda I)\det(-\lambda I)
\end{equation*}
and 
\begin{equation*}
    \det \begin{pmatrix}
        -\lambda I & 0 \\
        B&BA-\lambda I
    \end{pmatrix} = \det(-\lambda I)\det(BA-\lambda I)
\end{equation*}

Finally, we show that $\det(BA-\lambda I) = \det(AB-\lambda I)$. Equivalently, that $AB$ and $BA$ have the same eigenvalues. Suppose $\lambda$ is an eigenvalue of $AB$ with eigenvector $x$. Then $AB x= \lambda x \iff BAB x= B(\lambda x)$. Then letting $y=Bx$, we have that $BA y = \lambda y$, so $\lambda$ is an eigenvalue of $BA$. The characteristic polynomials are necessarily the same, so $\det(BA-\lambda I) = \det(AB-\lambda I)$, which completes the proof. \\
% Finally, we show that $\det(BA-\lambda I) = \det(AB-\lambda I)$. Consider the fact that since $\det(AB) = \det(A)\det(B) = \det(B)\det(A) = \det(BA)$. Therefore let, 
% \begin{equation*}
%     M=\begin{pmatrix}
%         I&A\\
%         B&I
%     \end{pmatrix} \ \
%     N=\begin{pmatrix}
%         I&0\\
%         -B&I
%     \end{pmatrix}
% \end{equation*}

% Then 
% \begin{equation*}
%     MN = \begin{pmatrix}
%         I-AB & A\\
%         0&I 
%     \end{pmatrix} \ \ 
%     NM = \begin{pmatrix}
%         I&A\\
%         0&I-BA
%     \end{pmatrix}
% \end{equation*}
% Since $\det(MN) = \det(NM)$ and using our lemma, we have that $\det(BA-\lambda I) = \det(AB-\lambda I)$. \\

\textbf{Question 3.} First, note that since $\det A = \det A^T$, we have that $\det((A-\lambda I)^T) = \det(A^T - \lambda I)=\det(A - \lambda I)$, so the eigenvalues of $A$ and $A^T$ are the same. Therefore, the Gersgorin circles with radii defined by $r_i=\sum_{i=1}^m |a_{ij}|, i\neq j$ of $A^T$ (columns of $A$) 

contain all eigenvalues of $A$. These circles are centered at $a_{ii}=a^T_{ii}$. Equivalently, the theorem holds with column sums of $A$.\\

\textbf{Question 4.} First, consider the absolute row sums given by $r_{1,2,3,4} = 0.8, 0.1, 0.4, 0.1$. Then since we showed the Gershgorin discs can also be found by considering the absolute column sums, consider the column sums of columns 2 and 4, given by $c_{2,4} = 0.1, 0.1$. Therefore, the radius of each circle is $0.1$. Additionally, since $k+0.1<(k+1)-0.1$ the circles are disjoint, and we can conclude that there is exactly one eigenvalue in $|z-k| < 0.1$ for $k=1,2,3,4$. \\

\textbf{Question 5.} Lemma: If $Ay = \lambda y $, then $A^n y = \lambda^n y$. Proof (by induction on $n$). $n=1$ is handled in the definition. Then suppose $A^n y = \lambda^n y$ and show that $A^{n+1}y= \lambda^{n+1}y$. Then $A^{n+1}=AA^n = A(\lambda^n y) = \lambda^n (Ay) = \lambda^n \lambda = \lambda^{n+1}$. Then we have that $y^T A^k y = y^T y\lambda^k$, so 
\begin{equation*}
    \lim_{k \ra \infty} \frac{y^T A^{k+1}y}{y^T A^k y} = \frac{y^T y \lambda^{k+1}}{y^T y \lambda^k} \lambda 
\end{equation*}

Is an eigenvalue of $A$. \\

Now note that since $A$ is non-defective, $A$ is diagonalizable and therefore has an eigenbasis. Then consider an arbitray vector $y =\sum_{i=1}^m a_i v_i$ where $v_i$ is the $i$th eigenvector with corresponding eigenvalue $\lambda_i$. Then from our Lemma, we have that 
\begin{align*}
    A^n y = \sum_{i=1}^m \lambda_i^na_i v_i
\end{align*}
And therefore 
\begin{align*}
    \frac{y^T A^{n+1}y}{y^TA^n y} &= \frac{(\sum_{i=1}^m a_iv_i^T)(\sum_{i=1}^m \lambda_i^{n+1}a_i v_i)}{(\sum_{i=1}^m a_iv_i^T)(\sum_{i=1}^m \lambda_i^{n}a_i v_i)} = \frac{\sum_{i=1}^m \lambda_i^{n+1}a_i v_i}{\sum_{i=1}^m \lambda_i^{n}a_i v_i}
\end{align*}

Since $A$ is positive definite, we know that the eigenvalues are positive ($A x = \lambda x \implies x^TAx = \|x\| \lambda > 0$). Let $\lambda_1 \geq ... \geq \lambda_m$. Therefore, 

\begin{align*}
    \frac{\sum_{i=1}^m \lambda_i^{n+1}a_i v_i}{\sum_{i=1}^m \lambda_i^{n}a_i v_i} &= \frac{\lambda_1 a_1v_1+...+\lambda_m \left(\frac{\lambda_m}{\lambda_1}^n\right)a_mv_m}{a_1v_1+\left(\frac{\lambda_2}{\lambda_1}\right)^n a_2v_2 + ... +\left(\frac{\lambda_m}{\lambda_1}\right)^n a_m v_m}
\end{align*}

Where we divide the numerator and denominator by $\lambda_1^n$. Note that since $p(A)=\lambda_1$, $\lambda_i / \lambda_1 < 1$ for $i=2,...,m$. Then we have that as 
$n \ra \infty$ the quotient converges to $\lambda_1$. \\

\textbf{Question 6.} Lemma: $p(A) \leq \|A\|$. Proof: We consider the proof with the 2-norm given by $\|A\|_2 = \sup_{\|x\| = 1}\|Ax\|_2$, since all norms are equivalent in a finite vector space. Let $p(A) = |\lambda|$, and let the corresponding eigenvector be $x$ with $\|x\| = 1$. Then $\|Ax\| = \|\lambda x\| = |\lambda|$. Now consider an arbitary unit vector $u$. By the Cauchy-Schwartz inequality, we have that $\|Au\| \leq \|A\| \|u\| = \|A\|$, therefore $\|A \| \geq \|A u\|$ for all vectors $u$. In particular, $\|A\| \geq \|Ax\| = |\lambda|$, so $p(A) \leq \|A\|$. \\

Now, consider the fact that since $A$ has nonnegative entries, $\sum_{j}^m a_{ij}=1=\|A\|_1$, the 1-norm of $A$. Therefore, $p(A) < 1$, or equivalently, no eigenvalue has an absolute value greater than one. \\

\textbf{Question 7.} 
\begin{itemize}
    \item[a.] Consider the SVD of $A$ to be $A=U\Sigma V^T$. Then since $A^T = V\Sigma U^T$, and $A^TA = V\Sigma^2 V^T = AA^T = U\Sigma^2U^T$, we have that $U=T$. Let $\sigma_i$ be the $i$th singular value of $A$. Since $\sigma_i = \sqrt{\lambda_i(A^TA)}$, $A = U\Sigma U^T$ and  $A^TA = U\Sigma^2 U^T$ by virtue of $A$ being normal, $\sigma_i = \sqrt{\sigma_i^2} = |\lambda_i|$. 
    \item[b.] Since $\|A\|_2 = \sqrt{p(A^TA)} = \sigma_{\max}(A)$ by definition, and we just showed that $\sigma_i = |\lambda_i|$, we have that $\|A\|_2 = |\lambda_i|$. 
\end{itemize}   

\end{document}
