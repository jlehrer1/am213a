\documentclass{article}
\usepackage[utf8]{inputenc}
\setlength{\parindent}{0pt} 
\usepackage{amssymb}
\usepackage{amsmath}
\newcommand{\N}{\mathbb{N}}
\newcommand{\Z}{\mathbb{Z}}
\newcommand{\Q}{\mathbb{Q}}
\newcommand{\R}{\mathbb{R}}
\newcommand{\C}{\mathbb{C}}
\newcommand{\ra}{\longrightarrow}

\title{Homework 2}
\date{}
\author{Julian Lehrer}
\begin{document}
\maketitle
\textbf{Question 1.} c

\textbf{Question 3.} Let $A$ be symmetric and positive-definite, so $x^*Ax > 0$. First, note that since $A$ is symmetrics, $a_{ii}=a_{ii}^*$, so $a_{ii}$ is real since the only way a complex number can be equal to it's conjugate transpose is if the imaginary part is zero. Then let $x=e_i$, the $i$th basis vector. So $e_i^* A e_i = e_i^T A e_i = a_{ii} > 0$, therefore the diagonals are both real and strictly postive. 

\textbf{Question 4.} 


\end{document}
